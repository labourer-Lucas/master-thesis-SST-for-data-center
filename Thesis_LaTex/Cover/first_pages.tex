%%%%%%%%%%%%%%%%%%%%%%%%%%%%%%%%%%%%%%%%%%%%%%%%%%%%%%%%%%%%%
%% Cover and Preface
%%%%%%%%%%%%%%%%%%%%%%%%%%%%%%%%%%%%%%%%%%%%%%%%%%%%%%%%%%%%%

\pagenumbering{Alph}

%Titelseite
\pagestyle{empty}
\begingroup
\leftskip=1cm\rightskip=1cm
\begin{center}
\phantom{u}
\vspace*{0.5cm}

\begin{center}\includegraphics[clip,width=4cm,keepaspectratio]{Cover/TUM_Logo_color}\par\end{center}

\vspace*{1.5cm}

\large{Chair of High-Power Converter Systems\\
       TUM School of Engineering and Design\\
       Technical University of Munich\\
	   Prof. Dr.sc. ETH Zürich Marcelo Lobo Heldwein}\\

\vspace*{2cm}

\LARGE{Title of Master Thesis}

\vspace*{1.0cm}

\large{Max Mustermann}\\

\vspace*{1.5cm}

\begin{center}\includegraphics[clip,width=5.5cm,keepaspectratio]{Cover/TUM_Wappen}\par\end{center}

\end{center}

\endgroup

\cleardoublepage

\hfill\includegraphics{Cover/TUM.pdf} \par

\vspace{3cm}
{\noindent\Huge\bfseries Title\par}
\vspace{0.5cm}
{\noindent\LARGE\bfseries Master Thesis\par}

\vfill

{
\noindent\large to gain the academic title\par
\vspace{1em}
\noindent Master of Science \\(M.Sc.)\par
\vspace{1em}
\noindent in the degree program Electrical Engineering and Information Technology\par
\vspace{1em}
\noindent at Technical University of Munich\par
\vspace{3cm}
\noindent

\begin{tabular}{@{}ll@{}}
 Submitted by & Max Mustermann\\
              & Matriculation Number: 123456\\
							& \url{max.mustermann@tum.de}\\
				      \hfill \\
	on          &  \today	\\
				      \hfill \\

Supervisor:& M.Sc. Erika Mustermann\\
           & Chair of High-Power Converter Systems\\
					 & Technical University of Munich\\
					 & \url{wei.tian@tum.de}\\
\end{tabular}

}

\clearpage

\vspace*{40mm}
\fontsize{19pt}{21pt}\selectfont
Declaration

\normalsize\selectfont
\vspace{13.2mm}

The work in this thesis is based on research carried out at the Chair of High-Power Converter Systems, Technical University of Munich (TUM) supervised by M.Sc. Erika Mustermann.  No part of this thesis has been submitted elsewhere for any other academic degree or qualification and it is all my own work unless referenced to the contrary in the text.

\vspace{18.1mm}
\rule[2.0mm]{\linewidth}{0.5pt}
Place, Date, Signature

\pagenumbering{Roman}
\pagestyle{fancy}
\setcounter{page}{0}
\setcounter{secnumdepth}{3}
\setcounter{tocdepth}{3}

\chapter*{Abstract}

Write a concise summary in 100 to 250 words maximum (half a page maximum for a thesis). You may write your abstract with the following guidelines: (i)~State the general motivation of your research field in one or maximum two sentences, e.g.\ “Recently, kites are being investigated to generate sustainable power with a lower material demand compared to conventional wind turbines.” (ii)~State the specific problem you are dealing with in one or maximum two sentences, e.g.\ “The automatic control of the kite is a major challenge.” (iii)~Briefly state how you propose to solve the problem, e.g.\ “This paper proposes to use cascaded controllers consisting of \dots” (iv)~Briefly state how you verified your idea and briefly state important results and possible/important limitations, e.g.\ “Simulations and experiments with a small-scale system demonstrated the validity and stability of the developed controllers. An efficiency of \dots was achieved.”---The abstract is only one paragraph. Avoid abbreviations if possible and do not use any references to other publications or to parts of this document. Write the abstract with the following in mind: The abstract serves anybody to decide if a work is relevant for his/her work. If the reader thinks that the abstract sounds relevant to him/her, he/she would then continue, and might read your conclusions right afterwards before reading all other sections. Consequently, abstract and conclusions are very important parts of your report/thesis.

\chapter*{Acknowledgments}

	\dots if you would like to thank someone. In a thesis, acknowledgements are usually put on one of the first pages.
	
\phantom{blabla}

\hspace*{\fill}Munich, in September 2022\\
\hspace*{\fill}Max Mustermann

\chapter*{Preface}

	\dots if you would like to write a preface.
	
\phantom{blabla}

\hspace*{\fill}Munich, in September 2022\\
\hspace*{\fill}Max Mustermann

\setlength{\parindent}{12pt}